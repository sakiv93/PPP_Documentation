\documentclass[11pt]{article}

\usepackage{amsmath}
\usepackage{amsfonts}
\usepackage{amssymb}
\usepackage{graphicx}
\usepackage[center]{caption}
\usepackage{mathtools}
\usepackage{lipsum}
\usepackage{stackengine}
\usepackage{fancyhdr}
\usepackage{caption}
\usepackage{tikz}
\usetikzlibrary{shapes.geometric, arrows}
\usepackage{float}
\usepackage[a4paper,left=1in,right=1in,top=1in,bottom=1in,footskip=.25in]{geometry}
\usepackage{etoolbox}
\usepackage[nottoc]{tocbibind}
\usepackage{tabu}
\usepackage{enumitem,kantlipsum}
\usepackage{verbatim}
\usepackage{hyperref}
\begin{document}


\section{Aim}
This test case is for checking the capability of the written Isogeometric analysis code with a linear elastic loading.
\section{Problem description} \label{2DPWLELPD}
A 2D plate is subjected to mechanical loading as shown in Figure(\ref{XYLoading}). The material used is PZT-PIC151 ceramics.
The movement of bottom edge AB is fixed in y-direction and left edge AC in x-direction. A displacement of 0.1 mm is given on the right edge BD.
\begin{figure}[H]
	\centering
	\begin{minipage}{.5\textwidth}
		\centering
		\includegraphics[width=0.8\linewidth]{2DPlate.png}
		\captionof{figure}{2D Plate}
		\label{2Dplate}
	\end{minipage}%
	\begin{minipage}{.5\textwidth}
		\centering
		\includegraphics[width=0.9\linewidth]{XYLoading.png}
		\captionof{figure}{2D Plate with loading}
		\label{XYLoading}
	\end{minipage}
\end{figure}


\end{document}